\documentclass[UTF8, 12pt]{ctexart}
\usepackage{geometry}
\geometry{margin=1.5cm, vmargin={0pt,1cm}}
\setlength{\topmargin}{-1cm}
\setlength{\paperheight}{29.7cm}
\setlength{\textheight}{25.3cm}

\usepackage{amsfonts}
\usepackage{amsmath}
\usepackage{amssymb}
\usepackage{amsthm}
\usepackage{enumerate}
\usepackage{graphicx}
\usepackage{multicol}
\usepackage{fancyhdr}
\usepackage{layout}
\usepackage{listings}
\usepackage{float, caption}

\pagestyle{fancy}
\fancyhead{}
\lhead{朱思威, 3220104624}
\chead{数据结构与算法项目作业:四则混合运算器}
\rhead{\today}

\begin{document}
\section{设计思路}
对四则运算表达式的处理主要分为两个部分:
\begin{itemize}
  \item 将类型为 \texttt{std::string} 的中缀表达式转换为由 token 组成的 
  \texttt{std::queue<std::string>}.
  \item 将转换后的队列中的 token 压入栈,并在试图压入运算符时推出栈顶元素并做相应运算,将结果
  重新压入栈中.
\end{itemize}
\section{具体实现}
头文件 \texttt{expression\_evaluator.h} 中实现了用于计算四则运算表达式的函数\newline
\texttt{ExpressionEvaluator::Evaluate}, 它被包裹在类 \texttt{ExpressionEvaluator} 中是
为了隐藏它调用的函数.\texttt{ExpressionEvaluator} 中也仅暴露了这一个公共函数.

\texttt{Evaluate} 中将中缀表达式转换为后缀表达式的函数是 \texttt{InfixToPostfix}, 
它作为类\newline \texttt{ExpressionEvaluator} 中的私有函数.
\subsection{\texttt{InfixToPostfix} 的实现}
合法的数字运算表达式可分为数字部分和运算符部分,每一个部分将被作为一个整体(即 token).
\texttt{InfixToPostfix} 会用一个字符 \texttt{c} 遍历整个表达式字符串,若 \texttt{c} 是
空格,则进行下一个字符的检测.重点则在于分别数字和运算符.
\subsubsection{数字部分}
显而易见的是如果 \texttt{c} 是数字,那么应该将其和相连的数字整体作为一个 token.
因此设置一个缓冲区,即代码中的 \texttt{buffer}.只要遍历到数字就将其放入缓冲区中.直到检测到
运算符,再把缓冲区作为一个 token 压入结果队列 \texttt{output}.

由于要考虑小数点和科学计数法,\texttt{e}、\texttt{E} 和小数点也被作为数字部分放入
缓冲区中.关键的问题就在于如何检测此时的小数点和 \texttt{e} 是合法的.

首先显然的是缓冲区没有数字时小数点和 \texttt{e} 的输入显然不合法.
此外本程序使用了五个判定变量来确保小数点和 \texttt{e} 的合法性,避免不合法的数
(形如 \verb|2.1.1|).具体来说:
\begin{itemize}
  \item \texttt{point\_is\_added} 用来判断小数点是否已在数字中添加.若为真则表示小数已被
  添加过,再检测到数字中有小数则丢出异常.初始值为 \texttt{false},
  在检测到小数点进入 \texttt{buffer} 后为 \texttt{true},在 \texttt{buffer} 被
  压入结果队列后重新赋值为 \texttt{false}.
  \item \texttt{e\_is\_added} 与上类似,用于确保 \texttt{e} 或 \texttt{E} 不被重复添加.
  \item \texttt{previous\_is\_digit} 用于确保小数点和 \texttt{e} 仅在数字后面被添加.
  检测形如 \verb|1.e| 的语法错误.初始值为 \texttt{false},检测到 \texttt{c} 为数字时
  赋值为 \texttt{true},否则赋值为 \texttt{false}.
  \item \texttt{previous\_is\_e} 与上类似,用于确保指数部分的负号仅在 \texttt{e} 后被添加.
  \item \texttt{next\_must\_be\_digit} 用于确保小数点和 \texttt{e} 后必须添加数字,默认
  为 \texttt{false},\texttt{c} 为小数点和 \texttt{e} 后赋值 \texttt{true},
  否则赋值为 \texttt{false}.
\end{itemize}
通过对以上五个判定变量做条件控制,并在数字被压入 \texttt{buffer} 后更新其状态,确保了科学
计数法的合法输入.
\subsubsection{运算符部分}
将运算符压入栈,如果将要压入的运算符优先级并不比栈顶的运算符优先级大,则先把栈中的运算符推出,
作为 token 压入结果队列,直到将压入的运算符优先级比栈顶的大.检测到 \texttt{)} 时则把 
\texttt{(} 之上的运算符全推出压进结果队列中.

至于对 \texttt{-} 是负号还是减号的判断,通过标识变量 \texttt{expect\_negative\_sign} 
判断.其为真是预期负号输入,将 \verb|-a| 转换为 \verb|(0-a)|.当检测到运算符后预期为负号
输入,反之则预期减号.初始预期负号

至于为何不将 \texttt{-a} 直接整体作为 token 压入队列.因为这不支持形如 \verb|-(a)| 负号后带
括号的表达式.
\subsection{\texttt{Evaluate} 的实现}
通过 \texttt{std::string} 提供的 \texttt{stod} 将数字类型的 token 转换为 \texttt{double} 
压入结果栈中,并在遍历到运算符时将最顶上的两个数字做相应运算,再压入栈中.
\section{结果分析}
\texttt{main.cc} 作为主函数,编译后即可产生四则运算计算器程序,按提示输入表达式后即可计算.

\texttt{test.cc} 作为测试函数,调用 \texttt{ExpressionEvaluator::Evaluate} 计算合法
和不合法样例,若合法样例报错或者不合法样例报错,均中断,否则运行成功:提示所有样例通过.

经测试:主函数正常运行,能正常报错,支持小数点、括号、和科学计数法.测试程序通过.
\end{document}