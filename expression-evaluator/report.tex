\documentclass[UTF8]{ctexart}
\usepackage{geometry, CJKutf8}
\geometry{margin=1.5cm, vmargin={0pt,1cm}}
\setlength{\topmargin}{-1cm}
\setlength{\paperheight}{29.7cm}
\setlength{\textheight}{25.3cm}

% useful packages.
\usepackage{amsfonts}
\usepackage{amsmath}
\usepackage{amssymb}
\usepackage{amsthm}
\usepackage{enumerate}
\usepackage{graphicx}
\usepackage{multicol}
\usepackage{fancyhdr}
\usepackage{layout}
\usepackage{listings}
\usepackage{float, caption}

\lstset{
    basicstyle=\ttfamily, basewidth=0.5em
}

% some common command
\newcommand{\dif}{\mathrm{d}}
\newcommand{\avg}[1]{\left\langle #1 \right\rangle}
\newcommand{\difFrac}[2]{\frac{\dif #1}{\dif #2}}
\newcommand{\pdfFrac}[2]{\frac{\partial #1}{\partial #2}}
\newcommand{\OFL}{\mathrm{OFL}}
\newcommand{\UFL}{\mathrm{UFL}}
\newcommand{\fl}{\mathrm{fl}}
\newcommand{\op}{\odot}
\newcommand{\Eabs}{E_{\mathrm{abs}}}
\newcommand{\Erel}{E_{\mathrm{rel}}}

\begin{document}

\pagestyle{fancy}
\fancyhead{}
\lhead{朱思威, 3220104624}
\chead{数据结构与算法项目作业:四则混合运算器}
\rhead{\today}

\section{设计思路}

四则混合运算器仅考虑字符 \verb|+ - * / . ( )| 空子符和数字字符 \verb|[0-9]| 的输入。故设置循环字符变量 \verb|c| 遍历整个表达式,并分别根据其类型进行不同的处理。

\subsection{预处理}

首先将原本的字符串类型的中缀表达式转换为字符串 token 数组组成的后缀表达式,将多个数字字符组成的一个数放入一个 token 中,将算符独立放在一个 token 中,这样就方便后续的计算。

采用的算法是 Shungting Yard 算法:创建一个存放数的队列和存放算符的栈,通过算符的优先级根据一定的规则将算符依次放入栈和队列中。具体的实现在头文件 \verb|infix_to_postfix.h| 中。

\subsubsection{关于负数的处理}

\verb|infix_to_postfix| 通过 Boolean 变量 \verb|expect_negative| 来判断当 \verb|c = '-'| 时,\verb|c| 代表的是操作符减号还是放在数前面的负号。如果检测到是负号,形如 \verb|-x|,则转换为 \verb|(0-x)| 来处理。

\subsubsection{关于小数的处理}

另一个 Boolean 变量 \verb|is_added| 用于判断数中的小数点是否已添加,如果其值为 \verb|true|,那么在遇到小数点后报错:多个小数点。

\subsection{计算}

将处理完的表达式进行计算:把其中的 token 一一压入另一个栈中。如果 token 是数字则直接压入不做任何操作;如果 token 是算符则首先检查粘中是否至少有两个数 token,没有则报错。然后将两个数字退出并做相应的计算,将计算结果再次压回栈中。具体的实现在头文件 \verb|express_evaluator.h| 中。

\subsection{主程序}

主程序调用 \verb|express_evaluator.h| 中的计算函数 \verb|EvaluateExpression| 来计算从终端传入的字符串,并抓取异常抛出。若无异常则给出计算结果。

\section{结果处理}

在当前目录下通过命令 \verb|make main| 即可输入表达式。如果输入的是合法表达式,那么直接给出结果;反之丢出异常提示。

命令 \verb|make test| 即可根据 \verb|express_evaluator.h| 生成一个测试文件。其中提供了各种不同的合法测试和不合法测试。如果出现合法测试报异常或不合法测试不报异常则测试程序中断,否则最后输出 \verb|All tests passed|。

\end{document}

%%% Local Variables: 
%%% mode: latex
%%% TeX-master: t
%%% End: 
