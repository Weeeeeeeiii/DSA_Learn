\documentclass[UTF8]{ctexart}
\usepackage{geometry, CJKutf8}
\geometry{margin=1.5cm, vmargin={0pt,1cm}}
\setlength{\topmargin}{-1cm}
\setlength{\paperheight}{29.7cm}
\setlength{\textheight}{25.3cm}

% useful packages.
\usepackage{amsfonts}
\usepackage{amsmath}
\usepackage{amssymb}
\usepackage{amsthm}
\usepackage{enumerate}
\usepackage{graphicx}
\usepackage{multicol}
\usepackage{fancyhdr}
\usepackage{layout}
\usepackage{listings}
\usepackage{float, caption}

\lstset{
    basicstyle=\ttfamily, basewidth=0.5em
}

% some common command
\newcommand{\dif}{\mathrm{d}}
\newcommand{\avg}[1]{\left\langle #1 \right\rangle}
\newcommand{\difFrac}[2]{\frac{\dif #1}{\dif #2}}
\newcommand{\pdfFrac}[2]{\frac{\partial #1}{\partial #2}}
\newcommand{\OFL}{\mathrm{OFL}}
\newcommand{\UFL}{\mathrm{UFL}}
\newcommand{\fl}{\mathrm{fl}}
\newcommand{\op}{\odot}
\newcommand{\Eabs}{E_{\mathrm{abs}}}
\newcommand{\Erel}{E_{\mathrm{rel}}}

\begin{document}

\pagestyle{fancy}
\fancyhead{}
\lhead{朱思威,3220104624}
\chead{数据结构与算法第五次作业}
\rhead{Nov.4th, 2024}

\section{对 remove 函数的修改}

原 \verb|remove| 函数使用了值的替换来删除元素,这会造成性能的浪费。于是我使用了节点的替换和删除来实现对元素的删除。

首先辅助方法 \verb|detachMin| 实现了找到并删除该树的最小节点并返回其指针的功能。具体来说,\verb|node| 用来存储当前的节点,\verb|minNode| 作为指针的引用用于寻找最小的节点的指针。

然后在 \verb|remove| 中调用 \verb|detachMin| 来删除右子树的最小元素并把它存储在 \verb|minNode| 中。随后利用 \verb|t| 作为父节点的 \verb|right| 指针的引用来让父节点指向 \verb|minNode|, 并通过之前创造的 \verb|t| 的备份 \verb|oldNode| 来恢复左子树和右子树的一致性。最后释放 \verb|oldNode| 的内存。

这样实现的 \verb|remove| 函数通过了节点的移动和删除避免了值的拷贝所造成的性能浪费。

\section{测试的结果}

测试结果一切正常。

我用 valgrind 进行测试,发现没有发生内存泄露。

\end{document}

%%% Local Variables: 
%%% mode: latex
%%% TeX-master: t
%%% End: 
