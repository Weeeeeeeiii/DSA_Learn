\documentclass[UTF8]{ctexart}
\usepackage{geometry, CJKutf8}
\geometry{margin=1.5cm, vmargin={0pt,1cm}}
\setlength{\topmargin}{-1cm}
\setlength{\paperheight}{29.7cm}
\setlength{\textheight}{25.3cm}

% useful packages.
\usepackage{amsfonts}
\usepackage{amsmath}
\usepackage{amssymb}
\usepackage{amsthm}
\usepackage{enumerate}
\usepackage{graphicx}
\usepackage{multicol}
\usepackage{fancyhdr}
\usepackage{layout}
\usepackage{listings}
\usepackage{float, caption}

\lstset{
    basicstyle=\ttfamily, basewidth=0.5em
}

% some common command
\newcommand{\dif}{\mathrm{d}}
\newcommand{\avg}[1]{\left\langle #1 \right\rangle}
\newcommand{\difFrac}[2]{\frac{\dif #1}{\dif #2}}
\newcommand{\pdfFrac}[2]{\frac{\partial #1}{\partial #2}}
\newcommand{\OFL}{\mathrm{OFL}}
\newcommand{\UFL}{\mathrm{UFL}}
\newcommand{\fl}{\mathrm{fl}}
\newcommand{\op}{\odot}
\newcommand{\Eabs}{E_{\mathrm{abs}}}
\newcommand{\Erel}{E_{\mathrm{rel}}}

\begin{document}

\pagestyle{fancy}
\fancyhead{}
\lhead{朱思威,3220104624}
\chead{数据结构与算法第五次作业}
\rhead{Nov.4th, 2024}

\section{对 remove 函数的修改}

原 \verb|remove| 函数使用了值的替换来删除元素,这会造成性能的浪费。于是我使用了节点的替换和删除来实现对元素的删除。

首先辅助方法 \verb|detachMin| 实现了找到并删除该树的最小节点并返回其指针的功能。具体来说,\verb|node| 用来存储当前的节点,\verb|minNode| 作为指针的引用用于寻找最小的节点的指针。

然后在 \verb|remove| 中调用 \verb|detachMin| 来删除右子树的最小元素并把它存储在 \verb|minNode| 中。随后利用 \verb|t| 作为父节点的 \verb|right| 指针的引用来让父节点指向 \verb|minNode|, 并通过之前创造的 \verb|t| 的备份 \verb|oldNode| 来恢复左子树和右子树的一致性。最后释放 \verb|oldNode| 的内存。

这样实现的 \verb|remove| 函数通过了节点的移动和删除避免了值的拷贝所造成的性能浪费。

\section{测试的结果}

我们测试了树的无参数初始化、拷贝构造函数、移动构造函数、查找最小最大元素、查找是否存在某项元素、删除某个元素、清空树以及异常处理,结果正常。

测试源代码如下:

\begin{verbatim}
        // 测试插入功能
        bst.insert(10);
        bst.insert(5);
        bst.insert(15);
        bst.insert(3);
        bst.insert(7);
        bst.insert(12);
        bst.insert(18);

        // 测试打印树结构
        std::cout << "Initial Tree:" << std::endl;
        bst.printTree();

        // 测试查找最小和最大元素
        std::cout << "Minimum element: " << bst.findMin() << std::endl;
        std::cout << "Maximum element: " << bst.findMax() << std::endl;

        // 测试 contains 功能
        std::cout << "Contains 7? " << (bst.contains(7) ? "Yes" : "No") << std::endl;
        std::cout << "Contains 20? " << (bst.contains(20) ? "Yes" : "No") << std::endl;

        // 测试删除功能
        bst.remove(7);
        std::cout << "Tree after removing 7:" << std::endl;
        bst.printTree();

        bst.remove(10);
        std::cout << "Tree after removing 10:" << std::endl;
        bst.printTree();

        // 测试清空树
        bst.makeEmpty();
        std::cout << "Tree after making empty:" << std::endl;
        bst.printTree();

        // 测试是否为空
        std::cout << "Is tree empty? " << (bst.isEmpty() ? "Yes" : "No") << std::endl;

        // 测试拷贝构造函数和赋值运算符
        BinarySearchTree<int> bst2;
        bst2.insert(1);
        bst2.insert(3);
        bst2.insert(2);

        BinarySearchTree<int> bst3(bst2);
        std::cout << "Copied Tree (bst3):" << std::endl;
        bst3.printTree();

        BinarySearchTree<int> bst4;
        bst4 = bst2;
        std::cout << "Assigned Tree (bst4):" << std::endl;
        bst4.printTree();

        // 测试移动构造函数和移动赋值运算符
        BinarySearchTree<int> bst5(std::move(bst2));
        std::cout << "Moved Tree (bst5):" << std::endl;
        bst5.printTree();

        BinarySearchTree<int> bst6;
        bst6 = std::move(bst5);
        std::cout << "Move Assigned Tree (bst6):" << std::endl;
        bst6.printTree();

        BinarySearchTree<int> bst7;
        bst7.findMax();  // 测试异常处理
\end{verbatim}

测试结果如下:

\begin{verbatim}
        Initial Tree:
        3
        5
        7
        10
        12
        15
        18
        Minimum element: 3
        Maximum element: 18
        Contains 7? Yes
        Contains 20? No
        Tree after removing 7:
        3
        5
        10
        12
        15
        18
        Tree after removing 10:
        3
        5
        12
        15
        18
        Tree after making empty:
        Empty tree
        Is tree empty? Yes
        Copied Tree (bst3):
        1
        2
        3
        Assigned Tree (bst4):
        1
        2
        3
        Moved Tree (bst5):
        1
        2
        3
        Move Assigned Tree (bst6):
        1
        2
        3
        terminate called after throwing an instance of 'UnderflowException'
\end{verbatim}

\end{document}

%%% Local Variables: 
%%% mode: latex
%%% TeX-master: t
%%% End: 
