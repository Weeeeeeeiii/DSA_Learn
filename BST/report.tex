\documentclass[UTF8, a4paper, 12pt]{ctexart}

\usepackage[left=2.4cm,right=2.2cm,top=2.8cm,bottom=2.2cm]{geometry}
\usepackage{xcolor}
%\usepackage{inconsolata}% would give a nice typewriter font
\newcommand{\Cpp}{C\texttt{++}}
\usepackage{listings}
\usepackage{fancyhdr}

\lstset{
  language     = C++,
  basicstyle   = \ttfamily,
  keywordstyle = \color{blue}\textbf,
  commentstyle = \color{gray},
  stringstyle  = \color{green!70!black},
  stringstyle  = \color{red},
  columns      = fullflexible,
  numbers      = left,
  numberstyle  = \scriptsize\sffamily\color{gray},
  %caption      = A hello world program in \Cpp,
  showstringspaces = false,
  float,
}

\pagestyle{fancy}
\fancyhead{}
\lhead{朱思威,3220104624}
\chead{数据结构与算法第六次作业}
\rhead{Nov.10, 2024}

\begin{document}
\section{remove 如何实现}

\subsection{原 BST remove 的实现}
公共接口 \verb|remove()| 通过调用内部的

\begin{verbatim}
    remove(const Comparable &x, BinaryNode *&t)
\end{verbatim}

来实现,其中 \verb|x| 是待删除的元素,\verb|t| 用来传入(子)树根节点指针的引用。

然后通过递归来来查找到待删除元素的节点指针。当然,如果该元素不存在,则直接返回。接下来分两种情况:

\begin{itemize}
    \item 该节点的左右子节点均存在;
    \item 该节点有一个子节点不存在。
\end{itemize}

较简单的一种情况是有一个子节点不存在,那么我们直接让 \verb|t| 指向子节点的另一子节点(若也不存在则 \verb|t| 为空指针),再释放原节点。参考代码如下:

\begin{lstlisting}
BinaryNode *oldNode = t;
t = (t->left != nullptr) ? t->left : t->right;
delete oldNode;
return;
\end{lstlisting}

比较复杂的情况是两个子节点均存在,可以借助 \verb|detachMin(BinaryNode *&t)| 函数,其中 \verb|t| 作为(子)树的根节点指针引用传入。这个函数会在子树中寻找到含最小元素的节点,并在移除这个节点,以其指针作为返回值,实现如下:

\begin{lstlisting}
BinaryNode *detachMin(BinaryNode *&t)
{
    if (t->left != nullptr)
    {
        BinaryNode *oldNode = detachMin(t->left);

        // Height servises the AVL tree
        t->height = 1 + std::max(getHeight(t->left), getHeight(t->right)); 

        return oldNode;
    }
    BinaryNode *detachNode = t;
    t = detachNode->right;
    return detachNode;
}
\end{lstlisting}

随后调用 \verb|detachMin| 来“拆除” \verb|t| 指向的节点的右子树中含最小元素的节点,并返回它作为原节点的替代。

\subsection{remove 后使用“旋转”使 tree 平衡}
首先是 \verb|remove| 后得调整部分节点的高度,然后通过高度差来对不平衡的节点进行平衡。\verb|remove| 后的实现如下:

\begin{lstlisting}
// Update height of node
t->height = 1 + std::max(getHeight(t->left), getHeight(t->right));

// Rotate to make tree balance
int balance = getBalance(t);
if (balance > 1 && getBalance(t->left) >= 0)
{
    rightRotate(t);
}
else if (balance > 1 && getBalance(t->left) < 0)
{
    leftRotate(t->left);
    rightRotate(t);
}
else if (balance < -1 && getBalance(t->right) <= 0)
{
    leftRotate(t);
}
else if (balance < -1 && getBalance(t->right) > 0)
{
    rightRotate(t->right);
    leftRotate(t);
}
else
    ;
\end{lstlisting}

这样就在删除后将树“旋转”至平衡,并更新节点高度。

\section{测试结果}

成功编译运行 \verb|test.cpp|,耗时:

\begin{verbatim}
real    0m1.376s
user    0m1.367s
sys     0m0.000s
\end{verbatim}

使用的 CPU 为 AMD Ryzen 5 5600G,开启 O2 优化。本机必须通过

\begin{verbatim}
$ ulimit -s unlimited
\end{verbatim}

取消系统栈空间限制才能正常运行,否则报字段错误。
\end{document}