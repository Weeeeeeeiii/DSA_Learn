\documentclass[UTF8, 12pt]{ctexart}
\usepackage{geometry}

\geometry{margin=1.5cm, vmargin={0pt,1cm}}
\setlength{\topmargin}{-1cm}
\setlength{\paperheight}{29.7cm}
\setlength{\textheight}{25.3cm}
\setlength{\parskip}{4pt}
\setlength{\parindent}{0pt}

\usepackage{amsfonts}
\usepackage{amsmath}
\usepackage{amssymb}
\usepackage{amsthm}
\usepackage{enumerate}
\usepackage{graphicx}
\usepackage{multicol}
\usepackage{fancyhdr}
\usepackage{layout}
\usepackage{listings}
\usepackage{float, caption}

\pagestyle{fancy}
\fancyhead{}
\lhead{朱思威, 3220104624}
\chead{数据结构与算法第七次作业:堆排序实现}
\rhead{\today}

\begin{document}
\section{设计思路}
关于 \verb|SortHeap| 函数,目的是对标 \verb|std::sort_heap|,
接收一个已成为最大堆的 \verb|vector| 并对它排序。

\verb|SortHeap| 会将接收的 \verb|vector| 的第一个元素(最大的元素)和最后一个元素替换,
然后忽略最后一个元素,将剩下的元素重新排序为最大堆。然后将第一个元素和倒数第二个元素替换,
忽略最后两个元素,将剩下的元素再次排序为最大堆。这样递归地将第一个元素和倒数第 $n$ 个元素替换,
并忽略最后 $n$ 个元素,将剩下的元素再排序为最大堆,直到 $n = (\text{vector 的长度}) - 1$。
具体实现细节:
\begin{verbatim}
auto length{vec.size}
for (auto index{length - 1}; index > 0; --index) {
  std::swap(vec[0], vec[index]);
  Heapify(vec, index); // 这步是递归地将剩下元素排序为最大堆,详见下面
}
\end{verbatim}
\verb|Heapify| 会从根节点开始,检查两个子节点是否大于父节点,如果是,挑选较大的子节点
和父节点交换。并递归地以被交换的节点作根节点重复上述流程,直到根节点成为叶子节点时停止。实现细节
如下:
\begin{verbatim}
template <typename T>
void Heapify(std::vector<T> &vec, typename std::vector<T>::size_type length,
             typename std::vector<T>::size_type index = 0) {
  auto left_child{index * 2 + 1};
  auto right_child{left_child + 1};
  auto largest{index};

  if (left_child < length && vec[left_child] > vec[largest]) {
    largest = left_child;
  }
  if (right_child < length && vec[right_child] > vec[largest]) {
    largest = right_child;
  }

  if (largest != index) {
    std::swap(vec[index], vec[largest]);
    Heapify(vec, length, largest);
  }
}
\end{verbatim}
\section{测试流程}
在 \verb|test.cpp| 中测试 \verb|SortHeap| 的两个方面:
\begin{itemize}
  \item 排序结果是否正确
  \item 与 \verb|std::sort_heap| 的效率差别
\end{itemize}
\verb|SortHeap| 与 \verb|std::sort_heap| 同样接受已排序为最大堆的 \verb|vector|,
分别调用两个函数进行排序,并对排序所用时间计时。
\section{测试结果}
分别采用长度为 1\,000\,000, 2\,000\,000, 5\,000\,000 的随机、有序、倒序、重复数组测试 
\verb|std::sort_heap| 和 \verb|SortHeap| 和排序的正确性和排序时间。在 \verb|O2| 优化下
结果如下:
\begin{table}[hp]
  \centering
  \begin{tabular}{|l|r|r|}
    \hline
    sequence type       & \verb|SortHeap| time & \verb|std::sort_heap| time \\ \hline
    random sequence     & 145 ms               & 91 ms                      \\ \hline
    ordered sequence    & 54 ms                & 31 ms                      \\ \hline
    reverse sequence    & 59 ms                & 38 ms                      \\ \hline
    repetitive sequence & 133 ms               & 88 ms                      \\ \hline
  \end{tabular}
  \caption{1\,000\,000 长度数组}
\end{table}
\begin{table}[hp]
  \centering
  \begin{tabular}{|l|r|r|}
    \hline
    sequence type       & \verb|SortHeap| time & \verb|std::sort_heap| time \\ \hline
    random sequence     & 366 ms               & 202 ms                     \\ \hline
    ordered sequence    & 114 ms               & 63 ms                      \\ \hline
    reverse sequence    & 137 ms               & 83 ms                      \\ \hline
    repetitive sequence & 393 ms               & 198 ms                     \\ \hline
  \end{tabular}
  \caption{2\,000\,000 长度数组}
\end{table}
\begin{table}[hp]
  \centering
  \begin{tabular}{|l|r|r|}
    \hline
    sequence type       & \verb|SortHeap| time & \verb|std::sort_heap| time \\ \hline
    random sequence     & 1987 ms              & 833 ms                     \\ \hline
    ordered sequence    & 304 ms               & 168 ms                     \\ \hline
    reverse sequence    & 364 ms               & 209 ms                     \\ \hline
    repetitive sequence & 1599 ms             & 741 ms                     \\ \hline
  \end{tabular}
  \caption{5\,000\,000 长度数组}
\end{table}
\end{document}